\documentclass[a4paper,12pt]{article}
\usepackage[margin=2cm]{geometry}
\title{ICTS-RRI Maths Circle, Saturday 13 January, 2024}
\author{Disha Kuzhively}
% \email{disha.jk@icts.res.in}
\date{\today}
\begin{document}
\maketitle
\emph{2:30 pm}
\section*{Divisibility}
The division algorithm is a fundamental mathematical concept that provides a systematic way to divide one integer by another, with the result expressed as a quotient and a remainder. The division algorithm ensures that when you divide $a$ by $b$, you can express $a$ as a multiple of $b$ plus a remainder, and the remainder is always less than the absolute value of the divisor $b$. We say that $b$ divides $a$ when the remainder is zero.

Tests of divisibility are rules or criteria that help determine whether a given number is divisible by another without performing the actual division. You may have already come across these in your school.

Let's consider a problem: What is the remainder when $2^{100}$ is divided by $101$?

We can solve such problems without actually performing the division. Here are some warm-up problems to help you get started.
\begin{enumerate}
    \item Check if the following numbers are divisible by 4:
    \begin{enumerate}
        \item 632,
        \item 7896,
        \item 245,
        \item 1584
    \end{enumerate}
    \item Check if the following numbers are divisible by 6:
    \begin{enumerate}
        \item 846,
        \item 729,
        \item 312,
        \item 990
    \end{enumerate}
    \item Check if the following numbers are divisible by 8:
        \begin{enumerate}
            \item 1248,
            \item 572,
            \item 896,
            \item 633
        \end{enumerate}
    \item Investigate the divisibility by 5 of the sum of the squares of the first 10 positive integers: $1^2 + 2^2 + 3^2 + \ldots + 10^2$.
    \item Find the smallest positive integer that is divisible by 8, consists of only the digits 1 and 0, and each digit appears at least once.
    \item Investigate the divisibility by 9 of the sum of the cubes of the first 5 positive integers: $1^3 + 2^3 + 3^3 + 4^3 + 5^3$.
    \item Determine the largest five-digit number that is divisible by 10, has all distinct digits, and the sum of its digits is 25.
\end{enumerate}

\emph{3:45 pm}: Break for refreshments

\emph{4:00 pm} onwards
\section*{Modular Arithmetic and Test for Divisibility by 7}
Let's take a step back and try to understand how these tests for divisibility can be constructed. One way to go about this is by examining the remainders when powers of 10, $10^0$, $10^1$, $10^2$, $\ldots$, are divided by a specific number.

Compute the remainders when the powers of 10 ($10^0$, $10^1$, $10^2$, $\ldots$) are divided by 3. Do you observe any patterns or repetitions in the remainders? Repeat this exercise for the number 7. Based on your observation, propose a divisibility test for determining whether a number is divisible by 7?

What can be said about the remainders when powers of a different number, say 8 ($8^0$, $8^1$, $8^2$, $\ldots$) are divided by 3 or 7? Do you notice any patterns?
\newpage
\section*{Explore Further}
\begin{enumerate}
    \item Is there a number that gives a remainder of 1 when divided by 3, remainder 2 when divided by 4, remainder of 3 when divided by 5, and a remainder of 4 when divided by 6?  
\item Show that every fourth Fibonacci number is divisible by 3.
\item Given $n$ is an integer not divisible by 3, prove that $n^2-1$ is always divisible by 3.
\item Check whether $1^{11} + 2^{11} + 3^{11} + \ldots + 10^{11}$ is divisible by 11. What can you say about the divisibility of $1^n + 2^n + 3^n + \ldots + (n-1)^n$ by $n$ where $n$ is any natural number?
\item Arrange the ten digits; 0, 1, 2, 3, 4, 5, 6, 7, 8, 9 to create a number that is divisible by all the numbers from 2 to 18. How many such arrangments can you make?
\end{enumerate}
\end{document}